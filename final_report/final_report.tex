\documentclass[13pt]{amsart}
\usepackage[margin=0.75in]{geometry} % see geometry.pdf on how to lay out the page. There's lots.
\geometry{a4paper} % or letter or a5paper or ... etc
% \geometry{landscape} % rotated page geometry
\usepackage{amsmath}
%\PassOptionsToPackage{pdfpagelabels=false}{hyperref}
%\usepackage[figure,figure*]{hypcap}
\usepackage[dvipsnames]{xcolor} 
\usepackage{graphicx}
\usepackage{float}
\usepackage{subcaption}
\usepackage{url}
%\usepackage[font=small,labelfont=bf]{caption} % Required for specifying captions to tables and figures

% See the ``Article customise'' template for come common customisations

\title{Term Project Proposal}
\author{Troy Raen}
%\date{} % delete this line to display the current date

%%--------------------------------------------------------------------------------------------------------
% MACROS
% \newcommand{\Msun}{\mathrm{M}_{\odot}} % Msun/h
% \newcommand{\hw}[1]{{\color{TealBlue}[HW #1]}}
\newcommand{\Q}[1]{{\color{gray}\textbf{#1}}}
\newcommand{\p}[2]{\vspace{5mm} \textbf{#1: }{\color{gray}\textbf{#2}}}

%--------------------------------------------------------------------------------------------------------
%%% BEGIN DOCUMENT
\begin{document}

\maketitle
%\tableofcontents



\section{Problem Outline}


\section{Data Analysis}


\begin{center}
\begin{tabular}{|l|c|}
\multicolumn{2}{c}{} \\ \hline 
 & Correlations with Redshift \\ \hline
redshift & 1.0 \\ \hline
tu & \textbf{0.4551} \\ \hline
tg & 0.1438 \\ \hline
tr & 0.2717 \\ \hline
ti & 0.3933 \\ \hline
tz & 0.4234 \\ \hline
ty & 0.4136 \\ \hline
u10 & 0.3932 \\ \hline
uerr10 & \textbf{0.518} \\ \hline
g10 & 0.1435 \\ \hline
gerr10 & 0.06651 \\ \hline
r10 & 0.2716 \\ \hline
rerr10 & 0.1222 \\ \hline
i10 & 0.3931 \\ \hline
ierr10 & 0.3822 \\ \hline
z10 & 0.423 \\ \hline
zerr10 & \textbf{0.4605} \\ \hline
y10 & 0.4109 \\ \hline
yerr10 & 0.45 \\ \hline
\end{tabular}
\end{center}


??? 
several runs with same sample size and pool abs(photz - specz)./(1+specz)

NMAD = 1.48* median(abs(photz - specz)./(1+specz))
out10 = sum(abs(photz - specz)./(1+specz) > 0.1)/N

Features to use? Just observed colors + 1 mag = 6 features.
How are errors computed? Not sure. Should I use them in algorithm? Not without checking, if they are calculated from true magnitudes, then no.


general algorithms vs specific codes
algorithms: 
	* neural net 
	* random forest
	* k nearest neighbor
	* XG boost (variant of decision trees) python implementation (possibly use 'AdaBoostM1' in matlab: https://www.mathworks.com/matlabcentral/answers/423851-is-there-any-implementation-of-xgboost-algorithm-for-decision-trees-in-matlab)
	* ANNz2
	relevance vector machine

???





To Do:
create data files with colors
run RF and NN for several sample sizes and plot


fitrensemble optimization:
Method Bag
NumLearningCycles 495
LearnRate NaN
MinLeafSize 1

\end{document}
